\chapter{Node API}
\label{chap:api:node}

\section{Client Library}
\label{sec:api:node:client}

HyperDex provides Node bindings under the module \code{hyperdex-client}.  This
library wraps the HyperDex C Client library and enables use of native Javascript
data types.

This module was re-introduced in HyperDex 1.2.0.

\subsection{Building the HyperDex Node.js Binding}
\label{sec:api:node:building}

The HyperDex Node.js Binding must be built and installed after HyperDex is built
and installed.  After installing HyperDex, you can build the Node.js bindings
from either source or git checkout with:

\begin{consolecode}
% cd bindings/node.js
% node rebuild
\end{consolecode}

\subsection{Using Node.js Within Your Application}
\label{sec:api:node:using}

All client operation are defined in the \code{hyperdex\_client} module.  You can
access this in your program with:

\begin{javascriptcode}
var hyperdex_client = require('hyperdex-client');
\end{javascriptcode}

\subsection{Hello World}
\label{sec:api:node:hello-world}

The following is a minimal application that stores the value "Hello World" and
then immediately retrieves the value:

\inputminted{javascript}{\topdir/node.js/client/hello-world.js}

You can run this example with:

\begin{consolecode}
% node hello-world.js
put: true
get: [object Object]
\end{consolecode}

Right away, there are several points worth noting in this example:

\begin{itemize}
\item Each operation takes a callback.  While the operation is outstanding, your
program is free to execute other code.

\item Javascript types are automatically converted to HyperDex types.  There's
no need to specify information such as the length of each string, as one would
do with the C API.

\item There's no need to manually enter the HyperDex event loop.  HyperDex will
add and remove itself from the event loop as operations start and finish.
\end{itemize}

\subsection{Asynchronous Operations}
\label{sec:api:node:async-ops}

HyperDex provides native integration with the asynchronous world of Node.js.
You can issue several operations concurrently, and Node.js and HyperDex will
work together to complete these operations quickly and efficiently.  It's easy
to work with data concurrently.  A common pattern is to keep a constant number
of operations outstanding concurrently:

\inputminted{javascript}{\topdir/node.js/client/window-pattern.js}

\subsection{Data Structures}
\label{sec:api:node:data-structures}

The Node bindings automatically manage conversion of data types from Javascript
to HyperDex types, enabling applications to be written in idiomatic Javascript.

\subsubsection{Examples}
\label{sec:api:node:examples}

This section shows examples of Java data structures that are recognized by
HyperDex.  The examples here are for illustration purposes and are not
exhaustive.

\paragraph{Strings}

The HyperDex client recognizes Javascript strings and buffers and automatically
converts them to HyperDex strings.  For example, the following two calls have
the same effect:

\begin{javascriptcode}
c.put("kv", "somekey", {v: "somevalue"}, function (success, err) {});
c.put("kv", "somekey", {v: new Buffer("somevalue")}, function (success, err) {});
\end{javascriptcode}

\paragraph{Integers}

The HyperDex client recognizes Javascript numbers and can convert them to
HyperDex integers.  For example:

\begin{javascriptcode}
c.put("kv", "somekey", {v: c.asInt(42)}, function (success, err) {});
\end{javascriptcode}

\paragraph{Floats}

The HyperDex client recognizes Javascript numbers and can convert them to
HyperDex floats.  For example:

\begin{javascriptcode}
c.put("kv", "somekey", {v: c.asFloat(3.1415)}, function (success, err) {});
\end{javascriptcode}

\paragraph{Lists}

The HyperDex client permits users to construct HyperDex lists from Javascript.
For example:

\begin{javascriptcode}
c.put("kv", "somekey", {v: c.asList(['a', 'b', 'c'])}, function (success, err) {});
c.put("kv", "somekey", {v: c.asList([1, 2, 3])}, function (success, err) {});
c.put("kv", "somekey", {v: c.asList([1.0, 0.5, 0.25])}, function (success, err) {});
\end{javascriptcode}

\paragraph{Sets}

The HyperDex client permits users to construct HyperDex sets from Javascript.
For example:

\begin{javascriptcode}
c.put("kv", "somekey", {v: c.asSet(['a', 'b', 'c'])}, function (success, err) {});
c.put("kv", "somekey", {v: c.asSet([1, 2, 3])}, function (success, err) {});
c.put("kv", "somekey", {v: c.asSet([1.0, 0.5, 0.25])}, function (success, err) {});
\end{javascriptcode}

\paragraph{Maps}

The HyperDex client permits users to construct HyperDex maps from Javascript.
For example:

\begin{javascriptcode}
c.put("kv", "somekey", {v1: c.asMap([["k", "v"]])}, function(success, err) {});
c.put("kv", "somekey", {v2: c.asMap([[1, 2]])}, function(success, err) {});
c.put("kv", "somekey", {v3: c.asMap([[3.14, 0.125]])}, function(success, err) {});
c.put("kv", "somekey", {v4: c.asMap([["a", 1]])}, function(success, err) {});
\end{javascriptcode}

\subsection{Attributes}
\label{sec:api:node:attributes}

Attributes in Node are specified in the form of a Javascript object.  As you can
see in the examples above, attributes are specified in the form:

\begin{javascriptcode}
{name: "value"}
\end{javascriptcode}

\subsection{Operations}
\label{sec:api:node:ops}

% Copyright (c) 2013-2014, Cornell University
% All rights reserved.
%
% Redistribution and use in source and binary forms, with or without
% modification, are permitted provided that the following conditions are met:
%
%     * Redistributions of source code must retain the above copyright notice,
%       this list of conditions and the following disclaimer.
%     * Redistributions in binary form must reproduce the above copyright
%       notice, this list of conditions and the following disclaimer in the
%       documentation and/or other materials provided with the distribution.
%     * Neither the name of HyperDex nor the names of its contributors may be
%       used to endorse or promote products derived from this software without
%       specific prior written permission.
%
% THIS SOFTWARE IS PROVIDED BY THE COPYRIGHT HOLDERS AND CONTRIBUTORS "AS IS"
% AND ANY EXPRESS OR IMPLIED WARRANTIES, INCLUDING, BUT NOT LIMITED TO, THE
% IMPLIED WARRANTIES OF MERCHANTABILITY AND FITNESS FOR A PARTICULAR PURPOSE ARE
% DISCLAIMED. IN NO EVENT SHALL THE COPYRIGHT OWNER OR CONTRIBUTORS BE LIABLE
% FOR ANY DIRECT, INDIRECT, INCIDENTAL, SPECIAL, EXEMPLARY, OR CONSEQUENTIAL
% DAMAGES (INCLUDING, BUT NOT LIMITED TO, PROCUREMENT OF SUBSTITUTE GOODS OR
% SERVICES; LOSS OF USE, DATA, OR PROFITS; OR BUSINESS INTERRUPTION) HOWEVER
% CAUSED AND ON ANY THEORY OF LIABILITY, WHETHER IN CONTRACT, STRICT LIABILITY,
% OR TORT (INCLUDING NEGLIGENCE OR OTHERWISE) ARISING IN ANY WAY OUT OF THE USE
% OF THIS SOFTWARE, EVEN IF ADVISED OF THE POSSIBILITY OF SUCH DAMAGE.

% This LaTeX file is generated by bindings/c.py

%%%%%%%%%%%%%%%%%%%% read_only %%%%%%%%%%%%%%%%%%%%
\pagebreak
\subsection{\code{read\_only}}
\label{api:c:read_only}
\index{read\_only!C API}
Change the cluster to and from read-only mode.


\paragraph{Definition:}
\begin{ccode}
int64_t hyperdex_admin_read_only(struct hyperdex_admin* admin,
        int ro,
        enum hyperdex_admin_returncode* status);
\end{ccode}

\paragraph{Parameters:}
\begin{itemize}[noitemsep]
\item \code{struct hyperdex\_admin* admin}\\
The HyperDex admin connection to use for the operation.

\item \code{int ro}\\
Flag indicating whether the cluster is set to read-only mode or read-write mode.
Set to non-zero for read-only mode, and zero for read-write mode.

\end{itemize}

\paragraph{Returns:}
\begin{itemize}[noitemsep]
\item \code{enum hyperdex\_admin\_returncode* status}\\
The status of the operation.  The admin library will fill in this variable
before returning this operation's request id from \code{hyperdex\_admin\_loop}.
The pointer must remain valid until the operation completes, and the pointer
should not be aliased to the status for any other outstanding operation.

\end{itemize}

%%%%%%%%%%%%%%%%%%%% wait_until_stable %%%%%%%%%%%%%%%%%%%%
\pagebreak
\subsection{\code{wait\_until\_stable}}
\label{api:c:wait_until_stable}
\index{wait\_until\_stable!C API}
Wait until the cluster is stable.  The cluster is considered stable when all
servers acknowledge the current configuration, and all initiated in previous
configurations complete.


\paragraph{Definition:}
\begin{ccode}
int64_t hyperdex_admin_wait_until_stable(struct hyperdex_admin* admin,
        enum hyperdex_admin_returncode* status);
\end{ccode}

\paragraph{Parameters:}
\begin{itemize}[noitemsep]
\item \code{struct hyperdex\_admin* admin}\\
The HyperDex admin connection to use for the operation.

\end{itemize}

\paragraph{Returns:}
\begin{itemize}[noitemsep]
\item \code{enum hyperdex\_admin\_returncode* status}\\
The status of the operation.  The admin library will fill in this variable
before returning this operation's request id from \code{hyperdex\_admin\_loop}.
The pointer must remain valid until the operation completes, and the pointer
should not be aliased to the status for any other outstanding operation.

\end{itemize}

%%%%%%%%%%%%%%%%%%%% fault_tolerance %%%%%%%%%%%%%%%%%%%%
\pagebreak
\subsection{\code{fault\_tolerance}}
\label{api:c:fault_tolerance}
\index{fault\_tolerance!C API}
Change the fault tolerance of \code{space} to the specified fault tolerance
threshold.


\paragraph{Definition:}
\begin{ccode}
int64_t hyperdex_admin_fault_tolerance(struct hyperdex_admin* admin,
        const char* space,
        uint64_t ft,
        enum hyperdex_admin_returncode* status);
\end{ccode}

\paragraph{Parameters:}
\begin{itemize}[noitemsep]
\item \code{struct hyperdex\_admin* admin}\\
The HyperDex admin connection to use for the operation.

\item \code{const char* space}\\
The name of the space as a string.

\item \code{uint64\_t ft}\\
The fault-tolerance threshold.  HyperDex will deploy at least one more server
than this threshold to ensure that this many servers may simultaneously fail
without introducing data loss or downtime.

\end{itemize}

\paragraph{Returns:}
\begin{itemize}[noitemsep]
\item \code{enum hyperdex\_admin\_returncode* status}\\
The status of the operation.  The admin library will fill in this variable
before returning this operation's request id from \code{hyperdex\_admin\_loop}.
The pointer must remain valid until the operation completes, and the pointer
should not be aliased to the status for any other outstanding operation.

\end{itemize}

%%%%%%%%%%%%%%%%%%%% validate_space %%%%%%%%%%%%%%%%%%%%
\pagebreak
\subsection{\code{validate\_space}}
\label{api:c:validate_space}
\index{validate\_space!C API}
Validate the provided space description.


\paragraph{Definition:}
\begin{ccode}
int hyperdex_admin_validate_space(struct hyperdex_admin* admin,
        const char* description,
        enum hyperdex_admin_returncode* status);
\end{ccode}

\paragraph{Parameters:}
\begin{itemize}[noitemsep]
\item \code{struct hyperdex\_admin* admin}\\
The HyperDex admin connection to use for the operation.

\item \code{const char* description}\\
A complete space description.

\end{itemize}

\paragraph{Returns:}
\begin{itemize}[noitemsep]
\item \code{enum hyperdex\_admin\_returncode* status}\\
The status of the operation.  The admin library will fill in this variable
before returning this operation's request id from \code{hyperdex\_admin\_loop}.
The pointer must remain valid until the operation completes, and the pointer
should not be aliased to the status for any other outstanding operation.

\end{itemize}

%%%%%%%%%%%%%%%%%%%% add_space %%%%%%%%%%%%%%%%%%%%
\pagebreak
\subsection{\code{add\_space}}
\label{api:c:add_space}
\index{add\_space!C API}
Add a new space to the cluster, using the provided space description.


\paragraph{Definition:}
\begin{ccode}
int64_t hyperdex_admin_add_space(struct hyperdex_admin* admin,
        const char* description,
        enum hyperdex_admin_returncode* status);
\end{ccode}

\paragraph{Parameters:}
\begin{itemize}[noitemsep]
\item \code{struct hyperdex\_admin* admin}\\
The HyperDex admin connection to use for the operation.

\item \code{const char* description}\\
A complete space description.

\end{itemize}

\paragraph{Returns:}
\begin{itemize}[noitemsep]
\item \code{enum hyperdex\_admin\_returncode* status}\\
The status of the operation.  The admin library will fill in this variable
before returning this operation's request id from \code{hyperdex\_admin\_loop}.
The pointer must remain valid until the operation completes, and the pointer
should not be aliased to the status for any other outstanding operation.

\end{itemize}

%%%%%%%%%%%%%%%%%%%% rm_space %%%%%%%%%%%%%%%%%%%%
\pagebreak
\subsection{\code{rm\_space}}
\label{api:c:rm_space}
\index{rm\_space!C API}
Remove \code{space} from the configuration and make it inaccessible to future
operations.


\paragraph{Definition:}
\begin{ccode}
int64_t hyperdex_admin_rm_space(struct hyperdex_admin* admin,
        const char* space,
        enum hyperdex_admin_returncode* status);
\end{ccode}

\paragraph{Parameters:}
\begin{itemize}[noitemsep]
\item \code{struct hyperdex\_admin* admin}\\
The HyperDex admin connection to use for the operation.

\item \code{const char* space}\\
The name of the space as a string.

\end{itemize}

\paragraph{Returns:}
\begin{itemize}[noitemsep]
\item \code{enum hyperdex\_admin\_returncode* status}\\
The status of the operation.  The admin library will fill in this variable
before returning this operation's request id from \code{hyperdex\_admin\_loop}.
The pointer must remain valid until the operation completes, and the pointer
should not be aliased to the status for any other outstanding operation.

\end{itemize}

%%%%%%%%%%%%%%%%%%%% mv_space %%%%%%%%%%%%%%%%%%%%
\pagebreak
\subsection{\code{mv\_space}}
\label{api:c:mv_space}
\index{mv\_space!C API}
Move space from name \code{source} to name \code{target}.  This is a metadata
operation and has the same cost as \code{add\_space}, even for populated spaces.


\paragraph{Definition:}
\begin{ccode}
int64_t hyperdex_admin_mv_space(struct hyperdex_admin* admin,
        const char* source,
        const char* target,
        enum hyperdex_admin_returncode* status);
\end{ccode}

\paragraph{Parameters:}
\begin{itemize}[noitemsep]
\item \code{struct hyperdex\_admin* admin}\\
The HyperDex admin connection to use for the operation.

\item \code{const char* source}\\
The name of the existing space.

\item \code{const char* target}\\
The new name for the space.

\end{itemize}

\paragraph{Returns:}
\begin{itemize}[noitemsep]
\item \code{enum hyperdex\_admin\_returncode* status}\\
The status of the operation.  The admin library will fill in this variable
before returning this operation's request id from \code{hyperdex\_admin\_loop}.
The pointer must remain valid until the operation completes, and the pointer
should not be aliased to the status for any other outstanding operation.

\end{itemize}

%%%%%%%%%%%%%%%%%%%% list_spaces %%%%%%%%%%%%%%%%%%%%
\pagebreak
\subsection{\code{list\_spaces}}
\label{api:c:list_spaces}
\index{list\_spaces!C API}
List all spaces held by the cluster.


\paragraph{Definition:}
\begin{ccode}
int64_t hyperdex_admin_list_spaces(struct hyperdex_admin* admin,
        enum hyperdex_admin_returncode* status,
        const char** spaces);
\end{ccode}

\paragraph{Parameters:}
\begin{itemize}[noitemsep]
\item \code{struct hyperdex\_admin* admin}\\
The HyperDex admin connection to use for the operation.

\end{itemize}

\paragraph{Returns:}
\begin{itemize}[noitemsep]
\item \code{enum hyperdex\_admin\_returncode* status}\\
The status of the operation.  The admin library will fill in this variable
before returning this operation's request id from \code{hyperdex\_admin\_loop}.
The pointer must remain valid until the operation completes, and the pointer
should not be aliased to the status for any other outstanding operation.

\item \code{const char** spaces}\\
A list of spaces created within the cluster.  The list is a multi-line C-string
with one space per line.  The C-string is valid until the next call into the
admin library, and will automatically be freed by the library.  It should not be
changed or freed by the library user.

\end{itemize}

%%%%%%%%%%%%%%%%%%%% list_indices %%%%%%%%%%%%%%%%%%%%
\pagebreak
\subsection{\code{list\_indices}}
\label{api:c:list_indices}
\index{list\_indices!C API}
XXX


\paragraph{Definition:}
\begin{ccode}
int64_t hyperdex_admin_list_indices(struct hyperdex_admin* admin,
        const char* space,
        enum hyperdex_admin_returncode* status,
        const char** indexes);
\end{ccode}

\paragraph{Parameters:}
\begin{itemize}[noitemsep]
\item \code{struct hyperdex\_admin* admin}\\
The HyperDex admin connection to use for the operation.

\item \code{const char* space}\\
The name of the space as a string.

\end{itemize}

\paragraph{Returns:}
\begin{itemize}[noitemsep]
\item \code{enum hyperdex\_admin\_returncode* status}\\
The status of the operation.  The admin library will fill in this variable
before returning this operation's request id from \code{hyperdex\_admin\_loop}.
The pointer must remain valid until the operation completes, and the pointer
should not be aliased to the status for any other outstanding operation.

\item \code{const char** indexes}\\
XXX

\end{itemize}

%%%%%%%%%%%%%%%%%%%% list_subspaces %%%%%%%%%%%%%%%%%%%%
\pagebreak
\subsection{\code{list\_subspaces}}
\label{api:c:list_subspaces}
\index{list\_subspaces!C API}
XXX


\paragraph{Definition:}
\begin{ccode}
int64_t hyperdex_admin_list_subspaces(struct hyperdex_admin* admin,
        const char* space,
        enum hyperdex_admin_returncode* status,
        const char** subspaces);
\end{ccode}

\paragraph{Parameters:}
\begin{itemize}[noitemsep]
\item \code{struct hyperdex\_admin* admin}\\
The HyperDex admin connection to use for the operation.

\item \code{const char* space}\\
The name of the space as a string.

\end{itemize}

\paragraph{Returns:}
\begin{itemize}[noitemsep]
\item \code{enum hyperdex\_admin\_returncode* status}\\
The status of the operation.  The admin library will fill in this variable
before returning this operation's request id from \code{hyperdex\_admin\_loop}.
The pointer must remain valid until the operation completes, and the pointer
should not be aliased to the status for any other outstanding operation.

\item \code{const char** subspaces}\\
XXX

\end{itemize}

%%%%%%%%%%%%%%%%%%%% add_index %%%%%%%%%%%%%%%%%%%%
\pagebreak
\subsection{\code{add\_index}}
\label{api:c:add_index}
\index{add\_index!C API}
Add a new index on a HyperDex attribute or document


\paragraph{Definition:}
\begin{ccode}
int64_t hyperdex_admin_add_index(struct hyperdex_admin* admin,
        const char* space,
        const char* attribute,
        enum hyperdex_admin_returncode* status);
\end{ccode}

\paragraph{Parameters:}
\begin{itemize}[noitemsep]
\item \code{struct hyperdex\_admin* admin}\\
The HyperDex admin connection to use for the operation.

\item \code{const char* space}\\
The name of the space as a string.

\item \code{const char* attribute}\\
The name of an attribute or document.

\end{itemize}

\paragraph{Returns:}
\begin{itemize}[noitemsep]
\item \code{enum hyperdex\_admin\_returncode* status}\\
The status of the operation.  The admin library will fill in this variable
before returning this operation's request id from \code{hyperdex\_admin\_loop}.
The pointer must remain valid until the operation completes, and the pointer
should not be aliased to the status for any other outstanding operation.

\end{itemize}

%%%%%%%%%%%%%%%%%%%% rm_index %%%%%%%%%%%%%%%%%%%%
\pagebreak
\subsection{\code{rm\_index}}
\label{api:c:rm_index}
\index{rm\_index!C API}
Remove an existing index


\paragraph{Definition:}
\begin{ccode}
int64_t hyperdex_admin_rm_index(struct hyperdex_admin* admin,
        uint64_t idxid,
        enum hyperdex_admin_returncode* status);
\end{ccode}

\paragraph{Parameters:}
\begin{itemize}[noitemsep]
\item \code{struct hyperdex\_admin* admin}\\
The HyperDex admin connection to use for the operation.

\item \code{uint64\_t idxid}\\
The index's unique identifier.  This will be an integer that identifies the
index across the lifetime of the cluster.

\end{itemize}

\paragraph{Returns:}
\begin{itemize}[noitemsep]
\item \code{enum hyperdex\_admin\_returncode* status}\\
The status of the operation.  The admin library will fill in this variable
before returning this operation's request id from \code{hyperdex\_admin\_loop}.
The pointer must remain valid until the operation completes, and the pointer
should not be aliased to the status for any other outstanding operation.

\end{itemize}

%%%%%%%%%%%%%%%%%%%% server_register %%%%%%%%%%%%%%%%%%%%
\pagebreak
\subsection{\code{server\_register}}
\label{api:c:server_register}
\index{server\_register!C API}
Manually register a new server with \code{token} bound to \code{address}.


\paragraph{Definition:}
\begin{ccode}
int64_t hyperdex_admin_server_register(struct hyperdex_admin* admin,
        uint64_t token,
        const char* address,
        enum hyperdex_admin_returncode* status);
\end{ccode}

\paragraph{Parameters:}
\begin{itemize}[noitemsep]
\item \code{struct hyperdex\_admin* admin}\\
The HyperDex admin connection to use for the operation.

\item \code{uint64\_t token}\\
The server's token which uniquely identifies the server.  This will be logged by
the server with the message prefix ``generated new random token'', and is
available via the administration tools that can introspect the cluster state.

\item \code{const char* address}\\
An IP/port pair to which the server is bound, specified as a human-readable
c-string.  For instance, \code{127.0.0.1:2012} specifies localhost port 2012.

\end{itemize}

\paragraph{Returns:}
\begin{itemize}[noitemsep]
\item \code{enum hyperdex\_admin\_returncode* status}\\
The status of the operation.  The admin library will fill in this variable
before returning this operation's request id from \code{hyperdex\_admin\_loop}.
The pointer must remain valid until the operation completes, and the pointer
should not be aliased to the status for any other outstanding operation.

\end{itemize}

%%%%%%%%%%%%%%%%%%%% server_online %%%%%%%%%%%%%%%%%%%%
\pagebreak
\subsection{\code{server\_online}}
\label{api:c:server_online}
\index{server\_online!C API}
Manually set server \code{token} to state \code{AVAILABLE}.


\paragraph{Definition:}
\begin{ccode}
int64_t hyperdex_admin_server_online(struct hyperdex_admin* admin,
        uint64_t token,
        enum hyperdex_admin_returncode* status);
\end{ccode}

\paragraph{Parameters:}
\begin{itemize}[noitemsep]
\item \code{struct hyperdex\_admin* admin}\\
The HyperDex admin connection to use for the operation.

\item \code{uint64\_t token}\\
The server's token which uniquely identifies the server.  This will be logged by
the server with the message prefix ``generated new random token'', and is
available via the administration tools that can introspect the cluster state.

\end{itemize}

\paragraph{Returns:}
\begin{itemize}[noitemsep]
\item \code{enum hyperdex\_admin\_returncode* status}\\
The status of the operation.  The admin library will fill in this variable
before returning this operation's request id from \code{hyperdex\_admin\_loop}.
The pointer must remain valid until the operation completes, and the pointer
should not be aliased to the status for any other outstanding operation.

\end{itemize}

%%%%%%%%%%%%%%%%%%%% server_offline %%%%%%%%%%%%%%%%%%%%
\pagebreak
\subsection{\code{server\_offline}}
\label{api:c:server_offline}
\index{server\_offline!C API}
Manually mark server \code{token} as \code{OFFLINE}.


\paragraph{Definition:}
\begin{ccode}
int64_t hyperdex_admin_server_offline(struct hyperdex_admin* admin,
        uint64_t token,
        enum hyperdex_admin_returncode* status);
\end{ccode}

\paragraph{Parameters:}
\begin{itemize}[noitemsep]
\item \code{struct hyperdex\_admin* admin}\\
The HyperDex admin connection to use for the operation.

\item \code{uint64\_t token}\\
The server's token which uniquely identifies the server.  This will be logged by
the server with the message prefix ``generated new random token'', and is
available via the administration tools that can introspect the cluster state.

\end{itemize}

\paragraph{Returns:}
\begin{itemize}[noitemsep]
\item \code{enum hyperdex\_admin\_returncode* status}\\
The status of the operation.  The admin library will fill in this variable
before returning this operation's request id from \code{hyperdex\_admin\_loop}.
The pointer must remain valid until the operation completes, and the pointer
should not be aliased to the status for any other outstanding operation.

\end{itemize}

%%%%%%%%%%%%%%%%%%%% server_forget %%%%%%%%%%%%%%%%%%%%
\pagebreak
\subsection{\code{server\_forget}}
\label{api:c:server_forget}
\index{server\_forget!C API}
Completely remove all state associated \code{token} from the cluster.


\paragraph{Definition:}
\begin{ccode}
int64_t hyperdex_admin_server_forget(struct hyperdex_admin* admin,
        uint64_t token,
        enum hyperdex_admin_returncode* status);
\end{ccode}

\paragraph{Parameters:}
\begin{itemize}[noitemsep]
\item \code{struct hyperdex\_admin* admin}\\
The HyperDex admin connection to use for the operation.

\item \code{uint64\_t token}\\
The server's token which uniquely identifies the server.  This will be logged by
the server with the message prefix ``generated new random token'', and is
available via the administration tools that can introspect the cluster state.

\end{itemize}

\paragraph{Returns:}
\begin{itemize}[noitemsep]
\item \code{enum hyperdex\_admin\_returncode* status}\\
The status of the operation.  The admin library will fill in this variable
before returning this operation's request id from \code{hyperdex\_admin\_loop}.
The pointer must remain valid until the operation completes, and the pointer
should not be aliased to the status for any other outstanding operation.

\end{itemize}

%%%%%%%%%%%%%%%%%%%% server_kill %%%%%%%%%%%%%%%%%%%%
\pagebreak
\subsection{\code{server\_kill}}
\label{api:c:server_kill}
\index{server\_kill!C API}
Manually change server \code{token} into the \code{KILLED} state.


\paragraph{Definition:}
\begin{ccode}
int64_t hyperdex_admin_server_kill(struct hyperdex_admin* admin,
        uint64_t token,
        enum hyperdex_admin_returncode* status);
\end{ccode}

\paragraph{Parameters:}
\begin{itemize}[noitemsep]
\item \code{struct hyperdex\_admin* admin}\\
The HyperDex admin connection to use for the operation.

\item \code{uint64\_t token}\\
The server's token which uniquely identifies the server.  This will be logged by
the server with the message prefix ``generated new random token'', and is
available via the administration tools that can introspect the cluster state.

\end{itemize}

\paragraph{Returns:}
\begin{itemize}[noitemsep]
\item \code{enum hyperdex\_admin\_returncode* status}\\
The status of the operation.  The admin library will fill in this variable
before returning this operation's request id from \code{hyperdex\_admin\_loop}.
The pointer must remain valid until the operation completes, and the pointer
should not be aliased to the status for any other outstanding operation.

\end{itemize}

%%%%%%%%%%%%%%%%%%%% backup %%%%%%%%%%%%%%%%%%%%
\pagebreak
\subsection{\code{backup}}
\label{api:c:backup}
\index{backup!C API}
Initiate a cluster-wide backup named \code{backup}.  This has the same behavior
as the \code{hyperdex backup} command discussed in Chapter~\ref{chap:backups}.


\paragraph{Definition:}
\begin{ccode}
int64_t hyperdex_admin_backup(struct hyperdex_admin* admin,
        const char* backup,
        enum hyperdex_admin_returncode* status,
        const char** backups);
\end{ccode}

\paragraph{Parameters:}
\begin{itemize}[noitemsep]
\item \code{struct hyperdex\_admin* admin}\\
The HyperDex admin connection to use for the operation.

\item \code{const char* backup}\\
The name for the backup created by this operation.

\end{itemize}

\paragraph{Returns:}
\begin{itemize}[noitemsep]
\item \code{enum hyperdex\_admin\_returncode* status}\\
The status of the operation.  The admin library will fill in this variable
before returning this operation's request id from \code{hyperdex\_admin\_loop}.
The pointer must remain valid until the operation completes, and the pointer
should not be aliased to the status for any other outstanding operation.

\item \code{const char** backups}\\
A list of servers that were backed up.  The list is a C-string with lines of the
form \code{"<token> <IP address> <path>"}.  The C-string is valid until the next
call into the admin library, and will automatically be freed by the library.  It
should not be changed or freed by the library user.

\end{itemize}

%%%%%%%%%%%%%%%%%%%% enable_perf_counters %%%%%%%%%%%%%%%%%%%%
\pagebreak
\subsection{\code{enable\_perf\_counters}}
\label{api:c:enable_perf_counters}
\index{enable\_perf\_counters!C API}
Start collecting performance counters from all servers in the cluster.


\paragraph{Definition:}
\begin{ccode}
int64_t hyperdex_admin_enable_perf_counters(struct hyperdex_admin* admin,
        enum hyperdex_admin_returncode* status,
        struct hyperdex_admin_perf_counter* pc);
\end{ccode}

\paragraph{Parameters:}
\begin{itemize}[noitemsep]
\item \code{struct hyperdex\_admin* admin}\\
The HyperDex admin connection to use for the operation.

\end{itemize}

\paragraph{Returns:}
\begin{itemize}[noitemsep]
\item \code{enum hyperdex\_admin\_returncode* status}\\
The status of the operation.  The admin library will fill in this variable
before returning this operation's request id from \code{hyperdex\_admin\_loop}.
The pointer must remain valid until the operation completes, and the pointer
should not be aliased to the status for any other outstanding operation.

\item \code{struct hyperdex\_admin\_perf\_counter* pc}\\
A struct where the admin library can store returned performance counters.
Memory regions returned via the struct will be valid until the next call into
the admin library and are managed by the library.  They should not be freed by
the user.

Only one performance counter may be enabled at a time.  Consecutive calls to
\code{enable\_perf\_counters} will return the same ID as the previous call, and
will return performance counters via the previously-used struct.  To change the
structure used for returning performance counters, first disable the performance
counters and then re-enable them.

\end{itemize}

%%%%%%%%%%%%%%%%%%%% disable_perf_counters %%%%%%%%%%%%%%%%%%%%
\pagebreak
\subsection{\code{disable\_perf\_counters}}
\label{api:c:disable_perf_counters}
\index{disable\_perf\_counters!C API}
Stop collecting performance counters from all servers in the cluster.


\paragraph{Definition:}
\begin{ccode}
void hyperdex_admin_disable_perf_counters(struct hyperdex_admin* admin);
\end{ccode}

\paragraph{Parameters:}
\begin{itemize}[noitemsep]
\item \code{struct hyperdex\_admin* admin}\\
The HyperDex admin connection to use for the operation.

\end{itemize}

\paragraph{Returns:}
Nothing
\pagebreak

\subsection{Working with Signals}
\label{sec:api:node:signals}

The HyperDex client module is signal-safe.  Should a signal interrupt the
client, it will raise an exception with status
\code{HYPERDEX\_CLIENT\_INTERRUPTED}.

\subsection{Working with Events}
\label{sec:api:node:threads}

The Node module naturally integrates with the Node.js event loop.  Each instance
of \code{Client} registers itself with the Node event loop and makes callbacks
as soon as events complete on the HyperDex side.

Put simply, a Node.js application can use \code{Client} instances in a
straight-forward fashion without worrying about threading or manual integration.
