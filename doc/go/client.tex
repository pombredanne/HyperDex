\chapter{Go Client API}
\label{chap:api:go-client}

HyperDex provide Go bindings to the client under the import path
\code{github.com/rescrv/HyperDex/bindings/go/client}.  This library wraps the C
client library and enables the use of native Go data types and uses Go's native
support for concurrency to exploit the asynchronous features of HyperDex.

\subsection{Using Go Within Your Application}
\label{sec:api:go-client:using}

All client operation are defined in the \code{client} package.  You can
access this in your program with:

\begin{gocode}
import ("github.com/rescrv/HyperDex/bindings/go/client")
\end{gocode}

In your go environment, run this to fetch the module into your \code{GOPATH}.

\begin{consolecode}
go get github.com/rescrv/HyperDex/bindings/go/client
\end{consolecode}

\subsection{Hello World}
\label{sec:api:go-client:hello-world}

The following is a minimal application that stores the value "Hello World" and
then immediately retrieves the value:

\inputminted{go}{\topdir/go/client/hello-world.go}

You can run this example with:

% XXX
\begin{consolecode}
% go run hello-world.go
put: "Hello World!"
got: map[v:Hello World!]
\end{consolecode}

In Go, every HyperDex operation appears to the application as a synchronous
call.  Operations will appear to block the calling go-routine and suspend
execution until the operation completes.  Behind the scenes, the client will use
the concurrency features of Go to allow many concurrent operations using the
same client library.  Additionally, Go types are automatically converted to
HyperDex's types under the hood.

\subsection{Data Structures}
\label{sec:api:go-client:data-structures}

The Go bindings automatically manage conversion of data types from Go to
HyperDex types, enabling applications to be written in idiomatic Go.

\subsubsection{Examples}
\label{sec:api:go-client:examples}

This section shows examples of Go data structures that are recognized by
HyperDex.  The examples here are for illustration purposes and are not
exhaustive.

\paragraph{Strings}

The HyperDex client recognizes Go's strings and automatically converts them to
HyperDex strings.  For example:

\begin{gocode}
err := client.Put("kv", "some key", client.Attributes{"v": "some value"})
\end{gocode}

\paragraph{Integers}

The HyperDex client recognizes Go's integers and automatically converts them to
HyperDex integers.  For example:

\begin{gocode}
err := client.Put("kv", "some key", client.Attributes{"v": 42})
err = client.Put("kv", "some key", client.Attributes{"v": int64(42)})
\end{gocode}

\paragraph{Floats}

The HyperDex client recognizes Go's floating point numbers and automatically
converts them to HyperDex floats.  For example:

\begin{gocode}
err := client.Put("kv", "some key", client.Attributes{"v": 3.14})
\end{gocode}

\paragraph{Lists}

The HyperDex client recognizes lists and automatically converts them to HyperDex
lists.  For example:

\begin{gocode}
err := client.Put("kv", "some key", client.Attributes{"v": client.List{"A", "B", "C"}})
\end{gocode}

\paragraph{Sets}

The HyperDex client recognizes sets and automatically converts them to HyperDex
sets.  For example:

\begin{gocode}
err := client.Put("kv", "some key", client.Attributes{"v": client.Set{"A", "B", "C"}})
\end{gocode}

\paragraph{Maps}

The HyperDex client recognizes Go dictionaries and automatically converts them
to HyperDex maps.  For example:

\begin{gocode}
err := client.Put("kv", "some key", client.Attributes{"v":
        client.Map{"A": "X", "B": "Y", "C": "Z"}})
\end{gocode}

\subsection{Attributes}
\label{sec:api:go-client:attributes}

Attributes in Go are provided as struct \code{client.Attributes}, which has the
same syntax as a Go map.  As you can see in the examples above, attributes are
specified in the form:

\begin{gocode}
client.Attributes{"v1": "A", "v2": 3.14}
\end{gocode}

\subsection{Map Attributes}
\label{sec:api:go-client:map-attributes}

Map attributes in Go are provided as struct \code{client.MapAttributes}, which
maps attribute names to \code{client.Map} objects that specify the key-value
pairs to manipulate.

\begin{gocode}
client.MapAttributes{"v": client.Map{"A": 3.14}}
\end{gocode}

\subsection{Predicates}
\label{sec:api:go-client:predicates}

Predicates in Go are specified as a slice of \code{client.Predicate}.  Here's a
variety of Predicates provided by Go:

\begin{gocode}
client.Predicate{"attr", "some value", client.EQUALS}
client.Predicate{"attr", 5, client.LESS_THAN}
client.Predicate{"attr", 5, client.LESS_EQUAL}
client.Predicate{"attr", 5, client.GREATER_THAN}
client.Predicate{"attr", 5, client.GREATER_EQUAL}
client.Predicate{"attr", "^prefix", client.REGEX}
client.Predicate{"attr", 42, client.LENGTH_EQUALS}
client.Predicate{"attr", 42, client.LENGTH_LESS_EQUALS}
client.Predicate{"attr", 42, client.LENGTH_GREATER_EQUALS}
client.Predicate{"attr", "needle", client.CONTAINS}
\end{gocode}

\subsection{Error Handling}
\label{sec:api:go-client:error-handling}

All error handling within the Go bindings is done via \code{client.Error}
returned from each operation.  If an error object is returned, its
status should be checked following the call.  For example, we may conditionally
change the value of \code{v} from \code{"foo"} to \code{"bar"} like this:

\begin{gocode}
err := client.CondPut("kv", "some key",
        []client.Predicate{{"v", "foo", client.EQUALS}},
        client.Attributes{"v": "bar"})
if err != nil {
    // it worked
} else if err.Status == client.CMPFAIL {
    // the existing value was not "foo"
} else if err.Status == client.NOTFOUND {
    // there is no existing value
} else {
    // something more fatal happened
    // this is an exceptional case
    // propagate the error or try again
}
\end{gocode}

\subsection{Operations}
\label{sec:api:go-client:ops}

% Copyright (c) 2013-2014, Cornell University
% All rights reserved.
%
% Redistribution and use in source and binary forms, with or without
% modification, are permitted provided that the following conditions are met:
%
%     * Redistributions of source code must retain the above copyright notice,
%       this list of conditions and the following disclaimer.
%     * Redistributions in binary form must reproduce the above copyright
%       notice, this list of conditions and the following disclaimer in the
%       documentation and/or other materials provided with the distribution.
%     * Neither the name of HyperDex nor the names of its contributors may be
%       used to endorse or promote products derived from this software without
%       specific prior written permission.
%
% THIS SOFTWARE IS PROVIDED BY THE COPYRIGHT HOLDERS AND CONTRIBUTORS "AS IS"
% AND ANY EXPRESS OR IMPLIED WARRANTIES, INCLUDING, BUT NOT LIMITED TO, THE
% IMPLIED WARRANTIES OF MERCHANTABILITY AND FITNESS FOR A PARTICULAR PURPOSE ARE
% DISCLAIMED. IN NO EVENT SHALL THE COPYRIGHT OWNER OR CONTRIBUTORS BE LIABLE
% FOR ANY DIRECT, INDIRECT, INCIDENTAL, SPECIAL, EXEMPLARY, OR CONSEQUENTIAL
% DAMAGES (INCLUDING, BUT NOT LIMITED TO, PROCUREMENT OF SUBSTITUTE GOODS OR
% SERVICES; LOSS OF USE, DATA, OR PROFITS; OR BUSINESS INTERRUPTION) HOWEVER
% CAUSED AND ON ANY THEORY OF LIABILITY, WHETHER IN CONTRACT, STRICT LIABILITY,
% OR TORT (INCLUDING NEGLIGENCE OR OTHERWISE) ARISING IN ANY WAY OUT OF THE USE
% OF THIS SOFTWARE, EVEN IF ADVISED OF THE POSSIBILITY OF SUCH DAMAGE.

% This LaTeX file is generated by bindings/c.py

%%%%%%%%%%%%%%%%%%%% read_only %%%%%%%%%%%%%%%%%%%%
\pagebreak
\subsection{\code{read\_only}}
\label{api:c:read_only}
\index{read\_only!C API}
Change the cluster to and from read-only mode.


\paragraph{Definition:}
\begin{ccode}
int64_t hyperdex_admin_read_only(struct hyperdex_admin* admin,
        int ro,
        enum hyperdex_admin_returncode* status);
\end{ccode}

\paragraph{Parameters:}
\begin{itemize}[noitemsep]
\item \code{struct hyperdex\_admin* admin}\\
The HyperDex admin connection to use for the operation.

\item \code{int ro}\\
Flag indicating whether the cluster is set to read-only mode or read-write mode.
Set to non-zero for read-only mode, and zero for read-write mode.

\end{itemize}

\paragraph{Returns:}
\begin{itemize}[noitemsep]
\item \code{enum hyperdex\_admin\_returncode* status}\\
The status of the operation.  The admin library will fill in this variable
before returning this operation's request id from \code{hyperdex\_admin\_loop}.
The pointer must remain valid until the operation completes, and the pointer
should not be aliased to the status for any other outstanding operation.

\end{itemize}

%%%%%%%%%%%%%%%%%%%% wait_until_stable %%%%%%%%%%%%%%%%%%%%
\pagebreak
\subsection{\code{wait\_until\_stable}}
\label{api:c:wait_until_stable}
\index{wait\_until\_stable!C API}
Wait until the cluster is stable.  The cluster is considered stable when all
servers acknowledge the current configuration, and all initiated in previous
configurations complete.


\paragraph{Definition:}
\begin{ccode}
int64_t hyperdex_admin_wait_until_stable(struct hyperdex_admin* admin,
        enum hyperdex_admin_returncode* status);
\end{ccode}

\paragraph{Parameters:}
\begin{itemize}[noitemsep]
\item \code{struct hyperdex\_admin* admin}\\
The HyperDex admin connection to use for the operation.

\end{itemize}

\paragraph{Returns:}
\begin{itemize}[noitemsep]
\item \code{enum hyperdex\_admin\_returncode* status}\\
The status of the operation.  The admin library will fill in this variable
before returning this operation's request id from \code{hyperdex\_admin\_loop}.
The pointer must remain valid until the operation completes, and the pointer
should not be aliased to the status for any other outstanding operation.

\end{itemize}

%%%%%%%%%%%%%%%%%%%% fault_tolerance %%%%%%%%%%%%%%%%%%%%
\pagebreak
\subsection{\code{fault\_tolerance}}
\label{api:c:fault_tolerance}
\index{fault\_tolerance!C API}
Change the fault tolerance of \code{space} to the specified fault tolerance
threshold.


\paragraph{Definition:}
\begin{ccode}
int64_t hyperdex_admin_fault_tolerance(struct hyperdex_admin* admin,
        const char* space,
        uint64_t ft,
        enum hyperdex_admin_returncode* status);
\end{ccode}

\paragraph{Parameters:}
\begin{itemize}[noitemsep]
\item \code{struct hyperdex\_admin* admin}\\
The HyperDex admin connection to use for the operation.

\item \code{const char* space}\\
The name of the space as a string.

\item \code{uint64\_t ft}\\
The fault-tolerance threshold.  HyperDex will deploy at least one more server
than this threshold to ensure that this many servers may simultaneously fail
without introducing data loss or downtime.

\end{itemize}

\paragraph{Returns:}
\begin{itemize}[noitemsep]
\item \code{enum hyperdex\_admin\_returncode* status}\\
The status of the operation.  The admin library will fill in this variable
before returning this operation's request id from \code{hyperdex\_admin\_loop}.
The pointer must remain valid until the operation completes, and the pointer
should not be aliased to the status for any other outstanding operation.

\end{itemize}

%%%%%%%%%%%%%%%%%%%% validate_space %%%%%%%%%%%%%%%%%%%%
\pagebreak
\subsection{\code{validate\_space}}
\label{api:c:validate_space}
\index{validate\_space!C API}
Validate the provided space description.


\paragraph{Definition:}
\begin{ccode}
int hyperdex_admin_validate_space(struct hyperdex_admin* admin,
        const char* description,
        enum hyperdex_admin_returncode* status);
\end{ccode}

\paragraph{Parameters:}
\begin{itemize}[noitemsep]
\item \code{struct hyperdex\_admin* admin}\\
The HyperDex admin connection to use for the operation.

\item \code{const char* description}\\
A complete space description.

\end{itemize}

\paragraph{Returns:}
\begin{itemize}[noitemsep]
\item \code{enum hyperdex\_admin\_returncode* status}\\
The status of the operation.  The admin library will fill in this variable
before returning this operation's request id from \code{hyperdex\_admin\_loop}.
The pointer must remain valid until the operation completes, and the pointer
should not be aliased to the status for any other outstanding operation.

\end{itemize}

%%%%%%%%%%%%%%%%%%%% add_space %%%%%%%%%%%%%%%%%%%%
\pagebreak
\subsection{\code{add\_space}}
\label{api:c:add_space}
\index{add\_space!C API}
Add a new space to the cluster, using the provided space description.


\paragraph{Definition:}
\begin{ccode}
int64_t hyperdex_admin_add_space(struct hyperdex_admin* admin,
        const char* description,
        enum hyperdex_admin_returncode* status);
\end{ccode}

\paragraph{Parameters:}
\begin{itemize}[noitemsep]
\item \code{struct hyperdex\_admin* admin}\\
The HyperDex admin connection to use for the operation.

\item \code{const char* description}\\
A complete space description.

\end{itemize}

\paragraph{Returns:}
\begin{itemize}[noitemsep]
\item \code{enum hyperdex\_admin\_returncode* status}\\
The status of the operation.  The admin library will fill in this variable
before returning this operation's request id from \code{hyperdex\_admin\_loop}.
The pointer must remain valid until the operation completes, and the pointer
should not be aliased to the status for any other outstanding operation.

\end{itemize}

%%%%%%%%%%%%%%%%%%%% rm_space %%%%%%%%%%%%%%%%%%%%
\pagebreak
\subsection{\code{rm\_space}}
\label{api:c:rm_space}
\index{rm\_space!C API}
Remove \code{space} from the configuration and make it inaccessible to future
operations.


\paragraph{Definition:}
\begin{ccode}
int64_t hyperdex_admin_rm_space(struct hyperdex_admin* admin,
        const char* space,
        enum hyperdex_admin_returncode* status);
\end{ccode}

\paragraph{Parameters:}
\begin{itemize}[noitemsep]
\item \code{struct hyperdex\_admin* admin}\\
The HyperDex admin connection to use for the operation.

\item \code{const char* space}\\
The name of the space as a string.

\end{itemize}

\paragraph{Returns:}
\begin{itemize}[noitemsep]
\item \code{enum hyperdex\_admin\_returncode* status}\\
The status of the operation.  The admin library will fill in this variable
before returning this operation's request id from \code{hyperdex\_admin\_loop}.
The pointer must remain valid until the operation completes, and the pointer
should not be aliased to the status for any other outstanding operation.

\end{itemize}

%%%%%%%%%%%%%%%%%%%% mv_space %%%%%%%%%%%%%%%%%%%%
\pagebreak
\subsection{\code{mv\_space}}
\label{api:c:mv_space}
\index{mv\_space!C API}
Move space from name \code{source} to name \code{target}.  This is a metadata
operation and has the same cost as \code{add\_space}, even for populated spaces.


\paragraph{Definition:}
\begin{ccode}
int64_t hyperdex_admin_mv_space(struct hyperdex_admin* admin,
        const char* source,
        const char* target,
        enum hyperdex_admin_returncode* status);
\end{ccode}

\paragraph{Parameters:}
\begin{itemize}[noitemsep]
\item \code{struct hyperdex\_admin* admin}\\
The HyperDex admin connection to use for the operation.

\item \code{const char* source}\\
The name of the existing space.

\item \code{const char* target}\\
The new name for the space.

\end{itemize}

\paragraph{Returns:}
\begin{itemize}[noitemsep]
\item \code{enum hyperdex\_admin\_returncode* status}\\
The status of the operation.  The admin library will fill in this variable
before returning this operation's request id from \code{hyperdex\_admin\_loop}.
The pointer must remain valid until the operation completes, and the pointer
should not be aliased to the status for any other outstanding operation.

\end{itemize}

%%%%%%%%%%%%%%%%%%%% list_spaces %%%%%%%%%%%%%%%%%%%%
\pagebreak
\subsection{\code{list\_spaces}}
\label{api:c:list_spaces}
\index{list\_spaces!C API}
List all spaces held by the cluster.


\paragraph{Definition:}
\begin{ccode}
int64_t hyperdex_admin_list_spaces(struct hyperdex_admin* admin,
        enum hyperdex_admin_returncode* status,
        const char** spaces);
\end{ccode}

\paragraph{Parameters:}
\begin{itemize}[noitemsep]
\item \code{struct hyperdex\_admin* admin}\\
The HyperDex admin connection to use for the operation.

\end{itemize}

\paragraph{Returns:}
\begin{itemize}[noitemsep]
\item \code{enum hyperdex\_admin\_returncode* status}\\
The status of the operation.  The admin library will fill in this variable
before returning this operation's request id from \code{hyperdex\_admin\_loop}.
The pointer must remain valid until the operation completes, and the pointer
should not be aliased to the status for any other outstanding operation.

\item \code{const char** spaces}\\
A list of spaces created within the cluster.  The list is a multi-line C-string
with one space per line.  The C-string is valid until the next call into the
admin library, and will automatically be freed by the library.  It should not be
changed or freed by the library user.

\end{itemize}

%%%%%%%%%%%%%%%%%%%% list_indices %%%%%%%%%%%%%%%%%%%%
\pagebreak
\subsection{\code{list\_indices}}
\label{api:c:list_indices}
\index{list\_indices!C API}
XXX


\paragraph{Definition:}
\begin{ccode}
int64_t hyperdex_admin_list_indices(struct hyperdex_admin* admin,
        const char* space,
        enum hyperdex_admin_returncode* status,
        const char** indexes);
\end{ccode}

\paragraph{Parameters:}
\begin{itemize}[noitemsep]
\item \code{struct hyperdex\_admin* admin}\\
The HyperDex admin connection to use for the operation.

\item \code{const char* space}\\
The name of the space as a string.

\end{itemize}

\paragraph{Returns:}
\begin{itemize}[noitemsep]
\item \code{enum hyperdex\_admin\_returncode* status}\\
The status of the operation.  The admin library will fill in this variable
before returning this operation's request id from \code{hyperdex\_admin\_loop}.
The pointer must remain valid until the operation completes, and the pointer
should not be aliased to the status for any other outstanding operation.

\item \code{const char** indexes}\\
XXX

\end{itemize}

%%%%%%%%%%%%%%%%%%%% list_subspaces %%%%%%%%%%%%%%%%%%%%
\pagebreak
\subsection{\code{list\_subspaces}}
\label{api:c:list_subspaces}
\index{list\_subspaces!C API}
XXX


\paragraph{Definition:}
\begin{ccode}
int64_t hyperdex_admin_list_subspaces(struct hyperdex_admin* admin,
        const char* space,
        enum hyperdex_admin_returncode* status,
        const char** subspaces);
\end{ccode}

\paragraph{Parameters:}
\begin{itemize}[noitemsep]
\item \code{struct hyperdex\_admin* admin}\\
The HyperDex admin connection to use for the operation.

\item \code{const char* space}\\
The name of the space as a string.

\end{itemize}

\paragraph{Returns:}
\begin{itemize}[noitemsep]
\item \code{enum hyperdex\_admin\_returncode* status}\\
The status of the operation.  The admin library will fill in this variable
before returning this operation's request id from \code{hyperdex\_admin\_loop}.
The pointer must remain valid until the operation completes, and the pointer
should not be aliased to the status for any other outstanding operation.

\item \code{const char** subspaces}\\
XXX

\end{itemize}

%%%%%%%%%%%%%%%%%%%% add_index %%%%%%%%%%%%%%%%%%%%
\pagebreak
\subsection{\code{add\_index}}
\label{api:c:add_index}
\index{add\_index!C API}
Add a new index on a HyperDex attribute or document


\paragraph{Definition:}
\begin{ccode}
int64_t hyperdex_admin_add_index(struct hyperdex_admin* admin,
        const char* space,
        const char* attribute,
        enum hyperdex_admin_returncode* status);
\end{ccode}

\paragraph{Parameters:}
\begin{itemize}[noitemsep]
\item \code{struct hyperdex\_admin* admin}\\
The HyperDex admin connection to use for the operation.

\item \code{const char* space}\\
The name of the space as a string.

\item \code{const char* attribute}\\
The name of an attribute or document.

\end{itemize}

\paragraph{Returns:}
\begin{itemize}[noitemsep]
\item \code{enum hyperdex\_admin\_returncode* status}\\
The status of the operation.  The admin library will fill in this variable
before returning this operation's request id from \code{hyperdex\_admin\_loop}.
The pointer must remain valid until the operation completes, and the pointer
should not be aliased to the status for any other outstanding operation.

\end{itemize}

%%%%%%%%%%%%%%%%%%%% rm_index %%%%%%%%%%%%%%%%%%%%
\pagebreak
\subsection{\code{rm\_index}}
\label{api:c:rm_index}
\index{rm\_index!C API}
Remove an existing index


\paragraph{Definition:}
\begin{ccode}
int64_t hyperdex_admin_rm_index(struct hyperdex_admin* admin,
        uint64_t idxid,
        enum hyperdex_admin_returncode* status);
\end{ccode}

\paragraph{Parameters:}
\begin{itemize}[noitemsep]
\item \code{struct hyperdex\_admin* admin}\\
The HyperDex admin connection to use for the operation.

\item \code{uint64\_t idxid}\\
The index's unique identifier.  This will be an integer that identifies the
index across the lifetime of the cluster.

\end{itemize}

\paragraph{Returns:}
\begin{itemize}[noitemsep]
\item \code{enum hyperdex\_admin\_returncode* status}\\
The status of the operation.  The admin library will fill in this variable
before returning this operation's request id from \code{hyperdex\_admin\_loop}.
The pointer must remain valid until the operation completes, and the pointer
should not be aliased to the status for any other outstanding operation.

\end{itemize}

%%%%%%%%%%%%%%%%%%%% server_register %%%%%%%%%%%%%%%%%%%%
\pagebreak
\subsection{\code{server\_register}}
\label{api:c:server_register}
\index{server\_register!C API}
Manually register a new server with \code{token} bound to \code{address}.


\paragraph{Definition:}
\begin{ccode}
int64_t hyperdex_admin_server_register(struct hyperdex_admin* admin,
        uint64_t token,
        const char* address,
        enum hyperdex_admin_returncode* status);
\end{ccode}

\paragraph{Parameters:}
\begin{itemize}[noitemsep]
\item \code{struct hyperdex\_admin* admin}\\
The HyperDex admin connection to use for the operation.

\item \code{uint64\_t token}\\
The server's token which uniquely identifies the server.  This will be logged by
the server with the message prefix ``generated new random token'', and is
available via the administration tools that can introspect the cluster state.

\item \code{const char* address}\\
An IP/port pair to which the server is bound, specified as a human-readable
c-string.  For instance, \code{127.0.0.1:2012} specifies localhost port 2012.

\end{itemize}

\paragraph{Returns:}
\begin{itemize}[noitemsep]
\item \code{enum hyperdex\_admin\_returncode* status}\\
The status of the operation.  The admin library will fill in this variable
before returning this operation's request id from \code{hyperdex\_admin\_loop}.
The pointer must remain valid until the operation completes, and the pointer
should not be aliased to the status for any other outstanding operation.

\end{itemize}

%%%%%%%%%%%%%%%%%%%% server_online %%%%%%%%%%%%%%%%%%%%
\pagebreak
\subsection{\code{server\_online}}
\label{api:c:server_online}
\index{server\_online!C API}
Manually set server \code{token} to state \code{AVAILABLE}.


\paragraph{Definition:}
\begin{ccode}
int64_t hyperdex_admin_server_online(struct hyperdex_admin* admin,
        uint64_t token,
        enum hyperdex_admin_returncode* status);
\end{ccode}

\paragraph{Parameters:}
\begin{itemize}[noitemsep]
\item \code{struct hyperdex\_admin* admin}\\
The HyperDex admin connection to use for the operation.

\item \code{uint64\_t token}\\
The server's token which uniquely identifies the server.  This will be logged by
the server with the message prefix ``generated new random token'', and is
available via the administration tools that can introspect the cluster state.

\end{itemize}

\paragraph{Returns:}
\begin{itemize}[noitemsep]
\item \code{enum hyperdex\_admin\_returncode* status}\\
The status of the operation.  The admin library will fill in this variable
before returning this operation's request id from \code{hyperdex\_admin\_loop}.
The pointer must remain valid until the operation completes, and the pointer
should not be aliased to the status for any other outstanding operation.

\end{itemize}

%%%%%%%%%%%%%%%%%%%% server_offline %%%%%%%%%%%%%%%%%%%%
\pagebreak
\subsection{\code{server\_offline}}
\label{api:c:server_offline}
\index{server\_offline!C API}
Manually mark server \code{token} as \code{OFFLINE}.


\paragraph{Definition:}
\begin{ccode}
int64_t hyperdex_admin_server_offline(struct hyperdex_admin* admin,
        uint64_t token,
        enum hyperdex_admin_returncode* status);
\end{ccode}

\paragraph{Parameters:}
\begin{itemize}[noitemsep]
\item \code{struct hyperdex\_admin* admin}\\
The HyperDex admin connection to use for the operation.

\item \code{uint64\_t token}\\
The server's token which uniquely identifies the server.  This will be logged by
the server with the message prefix ``generated new random token'', and is
available via the administration tools that can introspect the cluster state.

\end{itemize}

\paragraph{Returns:}
\begin{itemize}[noitemsep]
\item \code{enum hyperdex\_admin\_returncode* status}\\
The status of the operation.  The admin library will fill in this variable
before returning this operation's request id from \code{hyperdex\_admin\_loop}.
The pointer must remain valid until the operation completes, and the pointer
should not be aliased to the status for any other outstanding operation.

\end{itemize}

%%%%%%%%%%%%%%%%%%%% server_forget %%%%%%%%%%%%%%%%%%%%
\pagebreak
\subsection{\code{server\_forget}}
\label{api:c:server_forget}
\index{server\_forget!C API}
Completely remove all state associated \code{token} from the cluster.


\paragraph{Definition:}
\begin{ccode}
int64_t hyperdex_admin_server_forget(struct hyperdex_admin* admin,
        uint64_t token,
        enum hyperdex_admin_returncode* status);
\end{ccode}

\paragraph{Parameters:}
\begin{itemize}[noitemsep]
\item \code{struct hyperdex\_admin* admin}\\
The HyperDex admin connection to use for the operation.

\item \code{uint64\_t token}\\
The server's token which uniquely identifies the server.  This will be logged by
the server with the message prefix ``generated new random token'', and is
available via the administration tools that can introspect the cluster state.

\end{itemize}

\paragraph{Returns:}
\begin{itemize}[noitemsep]
\item \code{enum hyperdex\_admin\_returncode* status}\\
The status of the operation.  The admin library will fill in this variable
before returning this operation's request id from \code{hyperdex\_admin\_loop}.
The pointer must remain valid until the operation completes, and the pointer
should not be aliased to the status for any other outstanding operation.

\end{itemize}

%%%%%%%%%%%%%%%%%%%% server_kill %%%%%%%%%%%%%%%%%%%%
\pagebreak
\subsection{\code{server\_kill}}
\label{api:c:server_kill}
\index{server\_kill!C API}
Manually change server \code{token} into the \code{KILLED} state.


\paragraph{Definition:}
\begin{ccode}
int64_t hyperdex_admin_server_kill(struct hyperdex_admin* admin,
        uint64_t token,
        enum hyperdex_admin_returncode* status);
\end{ccode}

\paragraph{Parameters:}
\begin{itemize}[noitemsep]
\item \code{struct hyperdex\_admin* admin}\\
The HyperDex admin connection to use for the operation.

\item \code{uint64\_t token}\\
The server's token which uniquely identifies the server.  This will be logged by
the server with the message prefix ``generated new random token'', and is
available via the administration tools that can introspect the cluster state.

\end{itemize}

\paragraph{Returns:}
\begin{itemize}[noitemsep]
\item \code{enum hyperdex\_admin\_returncode* status}\\
The status of the operation.  The admin library will fill in this variable
before returning this operation's request id from \code{hyperdex\_admin\_loop}.
The pointer must remain valid until the operation completes, and the pointer
should not be aliased to the status for any other outstanding operation.

\end{itemize}

%%%%%%%%%%%%%%%%%%%% backup %%%%%%%%%%%%%%%%%%%%
\pagebreak
\subsection{\code{backup}}
\label{api:c:backup}
\index{backup!C API}
Initiate a cluster-wide backup named \code{backup}.  This has the same behavior
as the \code{hyperdex backup} command discussed in Chapter~\ref{chap:backups}.


\paragraph{Definition:}
\begin{ccode}
int64_t hyperdex_admin_backup(struct hyperdex_admin* admin,
        const char* backup,
        enum hyperdex_admin_returncode* status,
        const char** backups);
\end{ccode}

\paragraph{Parameters:}
\begin{itemize}[noitemsep]
\item \code{struct hyperdex\_admin* admin}\\
The HyperDex admin connection to use for the operation.

\item \code{const char* backup}\\
The name for the backup created by this operation.

\end{itemize}

\paragraph{Returns:}
\begin{itemize}[noitemsep]
\item \code{enum hyperdex\_admin\_returncode* status}\\
The status of the operation.  The admin library will fill in this variable
before returning this operation's request id from \code{hyperdex\_admin\_loop}.
The pointer must remain valid until the operation completes, and the pointer
should not be aliased to the status for any other outstanding operation.

\item \code{const char** backups}\\
A list of servers that were backed up.  The list is a C-string with lines of the
form \code{"<token> <IP address> <path>"}.  The C-string is valid until the next
call into the admin library, and will automatically be freed by the library.  It
should not be changed or freed by the library user.

\end{itemize}

%%%%%%%%%%%%%%%%%%%% enable_perf_counters %%%%%%%%%%%%%%%%%%%%
\pagebreak
\subsection{\code{enable\_perf\_counters}}
\label{api:c:enable_perf_counters}
\index{enable\_perf\_counters!C API}
Start collecting performance counters from all servers in the cluster.


\paragraph{Definition:}
\begin{ccode}
int64_t hyperdex_admin_enable_perf_counters(struct hyperdex_admin* admin,
        enum hyperdex_admin_returncode* status,
        struct hyperdex_admin_perf_counter* pc);
\end{ccode}

\paragraph{Parameters:}
\begin{itemize}[noitemsep]
\item \code{struct hyperdex\_admin* admin}\\
The HyperDex admin connection to use for the operation.

\end{itemize}

\paragraph{Returns:}
\begin{itemize}[noitemsep]
\item \code{enum hyperdex\_admin\_returncode* status}\\
The status of the operation.  The admin library will fill in this variable
before returning this operation's request id from \code{hyperdex\_admin\_loop}.
The pointer must remain valid until the operation completes, and the pointer
should not be aliased to the status for any other outstanding operation.

\item \code{struct hyperdex\_admin\_perf\_counter* pc}\\
A struct where the admin library can store returned performance counters.
Memory regions returned via the struct will be valid until the next call into
the admin library and are managed by the library.  They should not be freed by
the user.

Only one performance counter may be enabled at a time.  Consecutive calls to
\code{enable\_perf\_counters} will return the same ID as the previous call, and
will return performance counters via the previously-used struct.  To change the
structure used for returning performance counters, first disable the performance
counters and then re-enable them.

\end{itemize}

%%%%%%%%%%%%%%%%%%%% disable_perf_counters %%%%%%%%%%%%%%%%%%%%
\pagebreak
\subsection{\code{disable\_perf\_counters}}
\label{api:c:disable_perf_counters}
\index{disable\_perf\_counters!C API}
Stop collecting performance counters from all servers in the cluster.


\paragraph{Definition:}
\begin{ccode}
void hyperdex_admin_disable_perf_counters(struct hyperdex_admin* admin);
\end{ccode}

\paragraph{Parameters:}
\begin{itemize}[noitemsep]
\item \code{struct hyperdex\_admin* admin}\\
The HyperDex admin connection to use for the operation.

\end{itemize}

\paragraph{Returns:}
Nothing
